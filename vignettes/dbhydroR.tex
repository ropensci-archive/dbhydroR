\documentclass[12pt,notitlepage]{article}

%\usepackage{listings}
\usepackage[utf8]{inputenc}%jsta
\usepackage[english]{babel}%test
\usepackage{float}

\usepackage{natbib}
\usepackage[obeyspaces,spaces]{url}
\bibliographystyle{plainnat}
\bibpunct[; ]{(}{)}{,}{a}{}{;}

\usepackage{pifont,mdframed}%test

\usepackage{geometry}
\geometry{left=1.25in,right=1.25in,top=1.25in,bottom=1.25in}
\usepackage{rotating}
\usepackage{fancyhdr}
\usepackage[bookmarks,colorlinks,breaklinks,citecolor=red]{hyperref}
\usepackage{longtable}


%\usepackage{float}
\usepackage{graphicx,subfig}
% \usepackage{placeins}
\setlength\headheight{26pt}

\fancypagestyle{plain}{\fancyhf{}\fancyhead[R]{\includegraphics[width=6.0in,keepaspectratio=true]{sfwmd_bar8half_wordorexcel.png}}}

\author{Joseph Stachelek}
\title{dbhydroR: An R package to access the DBHYDRO Environmental Database}

%\VignetteIndexEntry{An R package to access the DBHYDRO Environmental Database}

\usepackage{Sweave}
\begin{document}
\Sconcordance{concordance:dbhydroR.tex:dbhydroR.Rnw:%
1 34 1 1 0 14 1 1 4 9 1 1 2 4 0 1 2 7 1 1 2 %
4 0 1 2 3 1 1 3 5 0 1 2 2 1 1 3 5 0 1 2 2 1 %
1 3 5 0 1 2 2 1 1 4 6 0 1 2 4 1 1 4 6 0 1 2 %
4 1 1 2 4 0 1 2 2 1 1 3 5 0 1 2 1 1 1 3 5 0 %
1 2 6 1 1 2 4 0 1 2 3 1 1 4 3 0 1 2 3 0 1 2 %
2 1 1 3 2 0 1 1 3 0 1 2 66 1}

\maketitle
%\tableofcontents
 


\section{Introduction}

This document introduces the \texttt{dbhydroR} package and its associated functions. These functions are aimed at improving programmatic workflows that query the DBHYDRO Environmental Database. HTTP requests are faciliated by the httr \citep{httr} package. 

\section{Package installation}

The \texttt{R} package \texttt{dbhydroR} is distributed via a \texttt{.tar.gz} (analagous to \texttt{.zip}) package archive file. This package contains the source code for package functions. In RStudio, it can be installed by navigating to \texttt{Tools} \verb|->| \texttt{Install Packages...} \verb|->| \texttt{Install from:} \verb|->| \texttt{Package Archive File}. Computers running the Windows operating system can only install binary \texttt{.zip} package archive files unless they have additional \href{https://cran.r-project.org/bin/windows/Rtools/}{compiler software} installed. The \texttt{dbhydroR} binary package can be installed by running the following commands from the \texttt{R} console:

\vspace{10pt}
\noindent\texttt{install.packages(c("httr", "RCurl", "reshape2", "XML"))}\\
\noindent\texttt{install.packages(paste(}\verb|"\\\\ad.sfwmd.gov\\DFSRoot\\data\\restoration_sciences",|\\
\verb|"\\projects\\joe_stachelek\\R\\dbhydroR_0.1-6.zip", sep = "")|\\
\texttt{, type = "win.binary", repos = NULL)}

\vspace{8pt}

\noindent Once installed, the package can be loaded using the following command:


\begin{Schunk}
\begin{Sinput}
 library(dbhydroR)
\end{Sinput}
\end{Schunk}


\section{Composing database queries}
\subsection{Water quality data}

Water quality data can be retrieved using the \texttt{getwq} function which takes four required arguments. The user must specify a station ID, a test name, and a date range. Station IDs can be located on the \href{http://my.sfwmd.gov/KMLEXT/CUSTOMKMLS/DBHydro/DBHydroKML/DBHYDRO_KML.kmz}{SFWMD Google Earth portal}. An abbreviated list of available test names can be found in the \nameref{sec:appendix} to this document while a full listing can be found at the \href{http://my.sfwmd.gov/dbhydroplsql/show_dbkey_info.show_data_type_info}{DBHYDRO metadata page}. Dates must be specified in YYYY-MM-DD format (e.g. 2015-02-26).   The following set of examples retrieve measurements between March 2011 and May 2012. They can be run from the R console by issuing the command:

\begin{Schunk}
\begin{Sinput}
 example(getwq)
\end{Sinput}
\end{Schunk}

\begin{itemize}
\item One variable at one station

\begin{Schunk}
\begin{Sinput}
 getwq(station_id = "FLAB08", date_min = "2011-03-01", 
       date_max = "2012-05-01", test_name = "CHLOROPHYLLA-SALINE")
\end{Sinput}
\end{Schunk}

\item One variable at multiple stations

\begin{Schunk}
\begin{Sinput}
 getwq(station_id = c("FLAB08","FLAB09"), date_min = "2011-03-01",
       date_max = "2012-05-01", test_name = "CHLOROPHYLLA-SALINE")
\end{Sinput}
\end{Schunk}

\item One variable at a wildcard station

\begin{Schunk}
\begin{Sinput}
 getwq(station_id = c("FLAB0%"), date_min = "2011-03-01", 
       date_max = "2012-05-01", test_name = "CHLOROPHYLLA-SALINE")
\end{Sinput}
\end{Schunk}

\item Multiple variables at multiple stations

\begin{Schunk}
\begin{Sinput}
 getwq(station_id = c("FLAB08","FLAB09"), date_min = "2011-03-01",
       date_max = "2012-05-01", test_name = c("CHLOROPHYLLA-SALINE",
       "SALINITY"))
\end{Sinput}
\end{Schunk}

\end{itemize}

\noindent By default, \verb|getwq| returns a \textit{cleaned output}. First, the cleaning function \verb|cleanwq| converts the raw output from native DBHYDRO \textit{long} format (each piece of data on its own row) to \textit{wide} format (each site x variable combination in its own column) using the reshape2 package \citep{reshape2}. Next, the extra columns associated with QA flags, LIMS, and District receiving are removed. Finally, row entries associated with QA \textit{blanks} are removed. Setting the \texttt{raw} flag to \texttt{TRUE} will force \verb|getwq| to retain this information. An example query that retains this information and the original \textit{long} formatting is shown below.

\begin{Schunk}
\begin{Sinput}
 getwq(station_id = "FLAB08", date_min = "2011-03-01", 
       date_max = "2011-05-01", test_name = "CHLOROPHYLLA-SALINE",
       raw = TRUE)
\end{Sinput}
\end{Schunk}

\subsection{Hydrologic data}

Hydrologic time series data can be retrieved using the \texttt{gethydro} function. The first task to accomplish prior to running \texttt{gethydro} is to identify one or more dbkeys. This can be done before-hand using the \texttt{getdbkey} function or the \href{http://my.sfwmd.gov/dbhydroplsql/show_dbkey_info.main_menu}{DBHYDRO Browser}. One useful strategy for finding desired dbkeys is to run the \texttt{getdbkey}   function interactively using progressively narrower search terms. For example, suppose we are interested in daily average wind data at Joe Bay but we have no alphanumeric \texttt{dbkey}. Initially we could run \texttt{getdbkey} with the \texttt{detail.level} set to "summary".

\begin{Schunk}
\begin{Sinput}
 getdbkey(stationid = "JBTS", category = "WEATHER", param = "WNDS",
          detail.level = "summary")
\end{Sinput}
\end{Schunk}

\noindent Our search returns two results but only one of them has a daily average (DA) measurement frequency. We can verify the remaining attributes of our likely dbkey by setting the \texttt{freq} parameter to "DA" and the \texttt{detail.level} parameter to "full".

\begin{Schunk}
\begin{Sinput}
 getdbkey(stationid = "JBTS", category = "WEATHER", param = "WNDS",
          freq = "DA", detail.level = "full")
\end{Sinput}
\end{Schunk}

\noindent This exact dbkey can only be returned reliably by specifying all of the \texttt{getdbkey} parameters applicable to the "WEATHER" category.

\begin{Schunk}
\begin{Sinput}
 getdbkey(stationid = "JBTS", category = "WEATHER", param = "WNDS",
          freq = "DA", stat = "MEAN", recorder = "CR10", agency = "WMD",
          detail.level = "dbkey")
\end{Sinput}
\end{Schunk}

\noindent Now that we have our dbkey in hand, we can use is as input to \texttt{gethydro}. In addition to a dbkey, we must specify a date range. Dates must be entered in YYYY-MM-DD format (e.g. 2015-02-26).

\begin{Schunk}
\begin{Sinput}
 gethydro(dbkey = "15081",
          date_min = "2013-01-01", date_max = "2013-02-02")
\end{Sinput}
\end{Schunk}

\noindent Alternatively, we can specify a set of arguments in our call to \texttt{gethydro} that will be passed to \texttt{getdbkey} on-the-fly. Use caution when using this strategy as complex stationid/category/parameter combinations can easily cause errors or return unexpected results. It is good practice to pre-screen your parameter values using \texttt{getdbkey}.

\begin{Schunk}
\begin{Sinput}
 gethydro(date_min = "2013-01-01", date_max = "2013-02-02",
          stationid = "JBTS", category = "WEATHER", param = "WNDS",
          freq = "DA", stat = "MEAN", recorder = "CR10", agency = "WMD")
\end{Sinput}
\end{Schunk}

\noindent The contents of multiple data streams can be returned by specifying multiple dbkeys or entering on-the-fly \texttt{getdbkey} queries that return multiple dbkeys.

\begin{Schunk}
\begin{Sinput}
 gethydro(dbkey = c("15081", "15069"), date_min = "2013-01-01",
          date_max = "2013-02-02")
\end{Sinput}
\end{Schunk}

\begin{Schunk}
\begin{Sinput}
 gethydro(date_min = "2013-01-01", date_max = "2013-02-02",
          category = "WEATHER", stationid = c("JBTS", "MBTS"),
          param = "WNDS", freq = "DA", stat = "MEAN")
\end{Sinput}
\end{Schunk}

\noindent More \texttt{gethydro} examples including queries of other \texttt{category} values ("SW", "GW", and "WQ") can be viewed by issuing the following commands from the \texttt{R} console:

\begin{Schunk}
\begin{Sinput}
 example(getdbkey)
 example(gethydro)
\end{Sinput}
\end{Schunk}

\newpage

\section{\label{sec:appendix}Appendix}
\subsection{Test names}
There are many test names available in DBHYDRO. A subset of these are detailed in the following table.\\

\begin{longtable}{| p{.45\textwidth} | p{.55\textwidth} |} 
\hline
Code\\
\hline
AMMONIA-N\\
CARBON, TOTAL ORGANIC\\
CHLOROPHYLL-A(LC)\\
CHLOROPHYLL-B(LC)\\
CHLOROPHYLLA-SALINE\\
DISSOLVED OXYGEN\\
KJELDAHL NITROGEN,TOTAL\\
NITRATE+NITRITE-N\\
NITRITE-N\\
PHEOPHYTIN-A(LC)\\
PHOSPHATE,ORTHO AS P\\
PHOSPHATE,TOTAL AS P\\
SALINITY\\
SILICA\\
SP CONDUCTIVITY, FIELD\\
TEMP\\
TOTAL NITROGEN\\
TURBIDITY\\
\hline
\end{longtable}
%\end{tabular}

\subsection{Further reading}
See section on URL-based data access in the \href{http://www.sfwmd.gov/portal/page/portal/xrepository/sfwmd_repository_pdf/dbhydrobrowseruserdocumentation.pdf}{DBHYDRO Browser User's Guide}

\medskip
 
 %\setlength{\bibsep}{0pt}
\bibliography{bib}

 
\end{document}
